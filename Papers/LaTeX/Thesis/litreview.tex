\section{Literature Review}\label{sect:litreview}

The use of Genetic Algorithms (GAs) for solving optimization problems is a well-established research area.
The first GA was proposed by Holland in 1975~\cite{holland1975adaptation} and it was based on the principles of natural selection and evolution.
The GA is a metaheuristic that is inspired by the biological evolution process and it is used to find optimal solutions to optimization problems.
It is also a stochastic population based algorithm that uses a population of candidate solutions to find the optimal solution. It is not guaranteed
to find a global optimum, but it is likely to find an acceptable solution that may be close to the global optimum.~\cite{alba2008introduction}

In the scope of optimizing resource allocation, GAs have been used to solve a plethora of problems. Ever since $2001$, the constant improvement and implementation
of the various parts of genetic algorithms, in particular on the crossover function, helped to improve the performances of GAs in the scheduling and
assignation of resources.~\cite{alcaraz2001robust}

During the last two decades, there have been many comparisons between GAs and other metaheuristics, such as Simulated Annealing (SA) and Tabu Search (TS).
%Cite this one up

The main difference between GAs and SA is that GAs are population based algorithms, while SA is a single solution based algorithm. The scope of the research
drifted towards the comparison between GAs and TS, which is a memory based algorithm~\cite{zolfaghari2002comparative} at first, then it shifted entirely
towards the comparison between GAs and Particle Swarm Optimization.~\cite{hassan2005comparison}

From there on, especially in the last decade, the focus of the research has been mostly on the improvement of the GAs themselves, in particular on the many parts comprising
the meta heuristic. 
From the point of view of the use of genetic algorithms for optimizing resource allocation in cloud computing environments,
the literature is rich and diverse.




to reduce energy consumption while maintaining or improving performance is an active area of research. 
While genetic algorithms have shown promising results in reducing energy consumption, there are still limitations to consider, 
such as computational expense, sensitivity to parameters, and the lack of guarantee of global optima. 
Ongoing research in this area is focused on addressing these limitations and developing more effective techniques 
for optimizing resource allocation in cloud computing environments using genetic algorithms. 
Therefore, the state-of-the-art regarding the use of genetic algorithms for saving energy in cloud computing is still evolving.